% !TEX root
% !TEX program = xelatex
% !BIB program = biber

\def \PrintMode{} %在使用电子版论文时,请将此行注释。在打印纸质论文时,请保持本行命令不被注释,然后打印时选择双面打印即可。

%用来控制是否启动打印模式的宏,请勿改动。
\ifx \PrintMode \undefined
    \def \SideMode{oneside}
    \def \ClearPageStyle{\clearpage}
\else
    \def \SideMode{twoside}
    \def \ClearPageStyle{\cleardoublepage}
\fi

\documentclass[a4paper,\SideMode,UTF8]{article} %A4纸,UTF-8

\usepackage[thmmarks,hyperref]{ntheorem} %定义命令环境使用的宏包
\usepackage[heading,zihao=-4]{ctex} %用来提供中文支持
\usepackage{amsmath,amssymb} %数学符号等相关宏包
\usepackage{graphicx} %插入图片所需宏包
\usepackage{xspace} %提供一些好用的空格命令
\usepackage{tikz-cd} %画交换图需要的宏包
\usepackage{url} %更好的超链接显示
\usepackage{array,booktabs} %表格相关的宏包
\usepackage{caption} %实现图片的多行说明
\usepackage{float} %图片与表格的更好排版
\usepackage{ulem} %更好的下划线
\usepackage[top=2.5cm, bottom=2.0cm, left=3.0cm, right=2.0cm]{geometry} %设置页边距

\usepackage{fontspec} %设置字体需要的宏包

%设置西文字体为Times New Roman,如果没有则以开源近似字体代替
\IfFontExistsTF{Times New Roman}{
	\setmainfont{Times New Roman}
}{
	\usepackage{newtxtext,newtxmath}
}

%设置文档中文字体。优先次序:中易 > Adobe > 华文(Mac) > Fandol
\IfFontExistsTF{SimSun}{
	\setCJKmainfont[AutoFakeBold=2,ItalicFont=KaiTi]{SimSun}
}{
	\IfFontExistsTF{AdobeSongStd-Light}{
		\setCJKmainfont[AutoFakeBold=2,ItalicFont=AdobeKaitiStd-Regular]{AdobeSongStd-Light}
	}{
		\IfFontExistsTF{STSong}{
			\setCJKmainfont[AutoFakeBold=2,BoldFont=STHeiti,ItalicFont=STKaiti]{STSong}
		}{
			\setCJKmainfont[AutoFakeBold=2,ItalicFont=FandolKai-Regular]{FandolSong-Regular}
		}
	}
}
\IfFontExistsTF{SimHei}{
	\setCJKsansfont[AutoFakeBold=2]{SimHei}
}{
	\IfFontExistsTF{AdobeHeitiStd-Regular}{
		\setCJKsansfont[AutoFakeBold=2]{AdobeHeitiStd-Regular}
	}{
		\IfFontExistsTF{STHeiti}{
			\setCJKsansfont [AutoFakeBold=2]{STHeiti}
		}{
			\setCJKsansfont[AutoFakeBold=2]{FandolHei-Regular}
		}
	}
}


\IfFileExists{zhlineskip.sty}{
	%Microsoft Word 样式的1.5倍行距(按中易字体计算)
	\usepackage[
		restoremathleading=false,
		UseMSWordMultipleLineSpacing,
		MSWordLineSpacingMultiple=1.5
	]{zhlineskip}
}{
	\linespread{1.621} %1.5倍行距
}

\showboxdepth=5
\showboxbreadth=5

%设置各级系统的编号格式
\setcounter{secnumdepth}{5}
\ctexset { section = { name={,.},number={\arabic{section}},format={\sffamily \zihao {-4}} } }
\ctexset { subsection = { name={,},number={\arabic{section}.\arabic{subsection}},format={\sffamily \zihao {-4}} } }
\ctexset { subsubsection = { name={,},number={\arabic{section}.\arabic{subsection}.\arabic{subsubsection}},format={\sffamily \zihao {-4}},indent=2em } }
\ctexset { paragraph = { name={,},number={\arabic{section}.\arabic{subsection}.\arabic{subsubsection}.\arabic{paragraph}},format={\sffamily \zihao {-4}},indent=4em } }
\ctexset { subparagraph = { name={,)},number={\arabic{subparagraph}},format={\sffamily \zihao {-4}},indent=6em } }

\usepackage[bottom,perpage]{footmisc}               %脚注,显示在每页底部,编号按页重置
\renewcommand*{\footnotelayout}{\zihao{-5}\rmfamily}  %设置脚注为小五号宋体
\renewcommand{\thefootnote}{\textcircled{\arabic{footnote}}}    %设置脚注标记为①,②,...

%设置页眉页脚
\usepackage{fancyhdr}
\lhead{华东师范大学学士学位论文}
\chead{}
\rhead{\TitleCHS}
\lfoot{}
\cfoot{\thepage}
\rfoot{}

\usepackage{xcolor} %彩色的文字

\usepackage[hidelinks]{hyperref} %各种超链接必备
\usepackage{cleveref} %交叉引用

%设置尾注
\usepackage{endnotes}
\renewcommand{\enotesize}{\zihao{-5}}
\renewcommand{\notesname}{\sffamily \zihao {-4} 尾注}
\renewcommand\enoteformat{
	\raggedright
	\leftskip=1.8em
	\makebox[0pt][r]{\theenmark. \rule{0pt}{\dimexpr\ht\strutbox+\baselineskip}}
}
\renewcommand\makeenmark{\textsuperscript{[尾注\theenmark]}}
\usepackage{footnotebackref}

%定义证明与解环境
\theoremstyle{nonumberplain}
\theorembodyfont{\upshape}
\theoremseparator{:}
\theoremsymbol{\ensuremath{\square}}
\newtheorem{proof}{\bfseries \sffamily 证明}
\theoremsymbol{\ensuremath{\blacksquare}}
\newtheorem{solution}{\bfseries \sffamily 解}

%定义各种常用环境
\theoremstyle{plain}
\theoremseparator{.}
\theorembodyfont{\upshape}
\theoremsymbol{}
\newtheorem{theorem}{\bfseries \sffamily 定理}[section]
\renewtheorem*{theorem*}{\bfseries \sffamily 定理}
\newtheorem{lemma}[theorem]{\bfseries \sffamily 引理}
\renewtheorem*{lemma*}{\bfseries \sffamily 引理}
\newtheorem{corollary}[theorem]{\bfseries \sffamily 推论}
\renewtheorem*{corollary*}{\bfseries \sffamily 推论}
\newtheorem{definition}[theorem]{\bfseries \sffamily 定义}
\renewtheorem*{definition*}{\bfseries \sffamily 定义}
\newtheorem{conjecture}[theorem]{\bfseries \sffamily 猜想}
\renewtheorem*{conjecture*}{\bfseries \sffamily 猜想}
\newtheorem{problem}[theorem]{\bfseries \sffamily 问题}
\renewtheorem*{problem*}{\bfseries \sffamily 问题}
\newtheorem{proposition}[theorem]{\bfseries \sffamily 命题}
\renewtheorem*{proposition*}{\bfseries \sffamily 命题}
\newtheorem{remark}[theorem]{\bfseries \sffamily 注记}
\renewtheorem*{remark*}{\bfseries \sffamily 注记}
\newtheorem{example}[theorem]{\bfseries \sffamily 例}
\renewtheorem*{example*}{\bfseries \sffamily 例}

%设置各种常用环境的交叉引用格式
\crefformat{theorem}{#2\bfseries{\sffamily 定理} #1#3}
\crefformat{lemma}{#2\bfseries{\sffamily 引理} #1#3}
\crefformat{corollary}{#2\bfseries{\sffamily 推论} #1#3}
\crefformat{definition}{#2\bfseries{\sffamily 定义} #1#3}
\crefformat{conjecture}{#2\bfseries{\sffamily 猜想} #1#3}
\crefformat{problem}{#2\bfseries{\sffamily 问题} #1#3}
\crefformat{proposition}{#2\bfseries{\sffamily 命题} #1#3}
\crefformat{remark}{#2\bfseries{\sffamily 注记} #1#3}
\crefformat{example}{#2\bfseries{\sffamily 例} #1#3}

%允许公式跨页显示
\allowdisplaybreaks

%屏蔽无关的Warning
\usepackage{silence}
\WarningFilter*{biblatex}{Conflicting options.\MessageBreak'eventdate=iso' requires 'seconds=true'.\MessageBreak Setting 'seconds=true'}

%使用biblatex管理文献,输出格式使用gb7714-2015标准,后端为biber
\usepackage[backend=biber,style=gb7714-2015,hyperref=true]{biblatex}
%将参考文献字体设置为五号
\renewcommand*{\bibfont}{\zihao{5}}

%生成感谢,请勿改动
\newcommand{\makeacknowledgement}{
	\clearpage
	\input{./ending/acknowledgement.tex}
}

%For Algorithm
\usepackage{algorithm,algorithmicx,algpseudocode}
\floatname{algorithm}{算法}
\renewcommand{\algorithmicrequire}{\textbf{输入:}}
\renewcommand{\algorithmicensure}{\textbf{输出:}}

%可能会需要在用自然语言描述算法步骤时使用的宏包
\usepackage{enumitem}

%表格单元格内换行
\newcommand{\tabincell}[2]{\begin{tabular}{@{}#1@{}}#2\end{tabular}}

%设置图、表的编号格式
\renewcommand{\thefigure}{\arabic{section}-\arabic{figure}}
\renewcommand{\thetable}{\arabic{section}-\arabic{table}}
%%每个section开始重置图、表的计数器
\makeatletter
\@addtoreset{table}{section}
\makeatother
\makeatletter
\@addtoreset{figure}{section}
\makeatother

%显示 1、2级标题
\setcounter{tocdepth}{2}

%设置目录字体
\usepackage{tocloft}
\renewcommand{\contentsname}{\centerline{目录}}
\renewcommand{\cftaftertoctitle}{\hfill}
\renewcommand{\cfttoctitlefont}{\sffamily \bfseries \zihao{-3}}
\renewcommand{\cftsubsubsecfont}{\rmfamily}
\renewcommand{\cftsubsecfont}{\rmfamily}
\renewcommand{\cftsecfont}{\rmfamily}
\renewcommand{\cftsecleader}{\cftdotfill{\cftdotsep}}
\renewcommand{\cftsecfont}{}
\renewcommand{\cftsecpagefont}{}

%灵活的行距定义(用于封面)
\usepackage{setspace}
 %加载各宏包以及本模板的主要设置
\addbibresource{./reference/thesis-ref.bib} %加载bib文件(参考文献)

\begin{document}

\pagestyle{empty} %不对正文前的各页面使用页眉页脚
\newgeometry{top=2.0cm, bottom=2.0cm,left=3.18cm, right=3.18cm} %设置用于首页的页边距

%请不要修改本页的任何代码!
%请不要修改本页的任何代码!
%请不要修改本页的任何代码!
\thispagestyle{empty}
\begin{titlepage}
	\captionsetup{belowskip=0pt}
	\input{paper_info.tex}
	\renewcommand{\ULthickness}{1.2pt}
	\begin{center}\noindent \bfseries \zihao{4}{\rmfamily{\CompleteYear 届本科生学士学位论文\hfill 学校代码:\uline{10269}}}\end{center}

	\begin{figure}[H]
		\centering
		\includegraphics{./figures/inner-cover(contains_font).eps}
	\end{figure}

	\vspace{-1em}
	\begin{spacing}{3}
		\centering
		\noindent\textbf{\zihao{1}{\rmfamily{\expandafter\uline\expandafter{\TitleCHS}}}}

		\noindent\textbf{\zihao{1}{\rmfamily{\expandafter\uline\expandafter{\TitleENG}}}}
	\end{spacing}

	\renewcommand{\ULthickness}{0.4pt}

	\begin{center}
		\vspace{-4em}
		\renewcommand{\arraystretch}{1.4
		}
		\bfseries\zihao{4}\rmfamily
		\begin{tabular}{ l r }
			姓\hfill 名:                   & \underline{{\makebox[6cm][c]{\Author}}}        \\
			学\hfill 号:                   & \underline{{\makebox[6cm][c]{\StudentID}}}     \\
			学\hfill 院:                   & \underline{{\makebox[6cm][c]{\Department}}}    \\
			专\hfill 业:                   & \underline{{\makebox[6cm][c]{\Major}}}         \\
			指\hfill 导\hfill 教\hfill 师: & \underline{{\makebox[6cm][c]{\Supervisor}}}    \\
			职\hfill 称:                   & \underline{{\makebox[6cm][c]{\AcademicTitle}}} \\
		\end{tabular}\\
		\vspace{1em}
		\CompleteYear\hspace*{1em}年\hspace*{1em}\CompleteMonth\hspace*{1em}月
	\end{center}
\end{titlepage} %插入内封面
\ClearPageStyle

\restoregeometry
%生成目录
\addtocontents{toc}{\protect\thispagestyle{empty}}
\begin{spacing}{1}
    \tableofcontents
\end{spacing}
\ClearPageStyle
\pagenumbering{Roman}
\thispagestyle{fancy}
\input{paper_info.tex}
\renewcommand\abstractname{\sffamily\zihao{-4} 摘要}
\phantomsection
\begin{abstract}
	\addcontentsline{toc}{section}{摘要}
	\zihao{5}\rmfamily
	\vspace{\baselineskip}
	\par 这里是中文摘要。这里是中文摘要。这里是中文摘要。这里是中文摘要。这里是中文摘要。这里是中文摘要。这里是中文摘要。这里是中文摘要。这里是中文摘要。这里是中文摘要。这里是中文摘要。这里是中文摘要。这里是中文摘要。这里是中文摘要。这里是中文摘要。这里是中文摘要。这里是中文摘要。这里是中文摘要。这里是中文摘要。
	\par 这里是中文摘要。这里是中文摘要。这里是中文摘要。这里是中文摘要。这里是中文摘要。这里是中文摘要。这里是中文摘要。这里是中文摘要。这里是中文摘要。这里是中文摘要。这里是中文摘要。这里是中文摘要。这里是中文摘要。这里是中文摘要。这里是中文摘要。这里是中文摘要。这里是中文摘要。这里是中文摘要。这里是中文摘要。
	\par 这里是中文摘要。这里是中文摘要。这里是中文摘要。这里是中文摘要。这里是中文摘要。这里是中文摘要。这里是中文摘要。这里是中文摘要。这里是中文摘要。这里是中文摘要。这里是中文摘要。这里是中文摘要。这里是中文摘要。这里是中文摘要。这里是中文摘要。这里是中文摘要。这里是中文摘要。这里是中文摘要。这里是中文摘要。
	\par 这里是中文摘要。这里是中文摘要。这里是中文摘要。这里是中文摘要。这里是中文摘要。这里是中文摘要。这里是中文摘要。这里是中文摘要。这里是中文摘要。这里是中文摘要。这里是中文摘要。这里是中文摘要。这里是中文摘要。这里是中文摘要。这里是中文摘要。这里是中文摘要。这里是中文摘要。这里是中文摘要。这里是中文摘要。
	\newline
	\newline
	{\bfseries \sffamily\zihao{5} 关键词:} \zihao{5}{\rmfamily \KeywordsCHS}
\end{abstract} %生成中英文摘要及关键词
\ClearPageStyle

\thispagestyle{fancy}
\renewcommand\abstractname{\zihao{-4} Abstract}
\phantomsection
\begin{abstract}
    \addcontentsline{toc}{section}{Abstract}
    \zihao{5}
    \par Here is Abstract in English. Here is Abstract in English. Here is Abstract in English. Here is Abstract in English. Here is Abstract in English. Here is Abstract in English. Here is Abstract in English. Here is Abstract in English. Here is Abstract in English. Here is Abstract in English. Here is Abstract in English. Here is Abstract in English. Here is Abstract in English. Here is Abstract in English. Here is Abstract in English. Here is Abstract in English. Here is Abstract in English. Here is Abstract in English. Here is Abstract in English. Here is Abstract in English. Here is Abstract in English. Here is Abstract in English. Here is Abstract in English. 
    \par Here is Abstract in English. Here is Abstract in English. Here is Abstract in English. Here is Abstract in English. Here is Abstract in English. Here is Abstract in English. Here is Abstract in English. Here is Abstract in English. Here is Abstract in English. Here is Abstract in English. Here is Abstract in English. Here is Abstract in English. Here is Abstract in English. Here is Abstract in English. Here is Abstract in English. Here is Abstract in English. Here is Abstract in English. Here is Abstract in English. Here is Abstract in English. Here is Abstract in English. Here is Abstract in English. Here is Abstract in English. Here is Abstract in English. 
    \par Here is Abstract in English. Here is Abstract in English. Here is Abstract in English. Here is Abstract in English. Here is Abstract in English. Here is Abstract in English. Here is Abstract in English. Here is Abstract in English. Here is Abstract in English. Here is Abstract in English. Here is Abstract in English. Here is Abstract in English. Here is Abstract in English. Here is Abstract in English. Here is Abstract in English. Here is Abstract in English. Here is Abstract in English. Here is Abstract in English. Here is Abstract in English. Here is Abstract in English. Here is Abstract in English. Here is Abstract in English. Here is Abstract in English.     
    \par Here is Abstract in English. Here is Abstract in English. Here is Abstract in English. Here is Abstract in English. Here is Abstract in English. Here is Abstract in English. Here is Abstract in English. Here is Abstract in English. Here is Abstract in English. Here is Abstract in English. Here is Abstract in English. Here is Abstract in English. Here is Abstract in English. Here is Abstract in English. Here is Abstract in English. Here is Abstract in English. Here is Abstract in English. Here is Abstract in English. Here is Abstract in English. Here is Abstract in English. Here is Abstract in English. Here is Abstract in English. Here is Abstract in English. 
    \par Here is Abstract in English. Here is Abstract in English. Here is Abstract in English. Here is Abstract in English. Here is Abstract in English. Here is Abstract in English. Here is Abstract in English. Here is Abstract in English. Here is Abstract in English. Here is Abstract in English. Here is Abstract in English. Here is Abstract in English. Here is Abstract in English. Here is Abstract in English. Here is Abstract in English. Here is Abstract in English. Here is Abstract in English. Here is Abstract in English. Here is Abstract in English. Here is Abstract in English. Here is Abstract in English. Here is Abstract in English. Here is Abstract in English. 
    \par Here is Abstract in English. Here is Abstract in English. Here is Abstract in English. Here is Abstract in English. Here is Abstract in English. Here is Abstract in English. Here is Abstract in English. Here is Abstract in English. Here is Abstract in English. Here is Abstract in English. Here is Abstract in English. Here is Abstract in English. Here is Abstract in English. Here is Abstract in English. Here is Abstract in English. Here is Abstract in English. Here is Abstract in English. Here is Abstract in English. Here is Abstract in English. Here is Abstract in English. Here is Abstract in English. Here is Abstract in English. Here is Abstract in English. 
    \par Here is Abstract in English. Here is Abstract in English. Here is Abstract in English. Here is Abstract in English. Here is Abstract in English. Here is Abstract in English. Here is Abstract in English. Here is Abstract in English. Here is Abstract in English. Here is Abstract in English. Here is Abstract in English. Here is Abstract in English. Here is Abstract in English. Here is Abstract in English. Here is Abstract in English. Here is Abstract in English. Here is Abstract in English. Here is Abstract in English. Here is Abstract in English. Here is Abstract in English. Here is Abstract in English. Here is Abstract in English. Here is Abstract in English. 
    \newline
    \newline
    {\bfseries \zihao{5} Keywords:} {\zihao{5} \KeywordsENG}
\end{abstract} %生成中英文摘要及关键词
\ClearPageStyle
\pagenumbering{arabic}
\pagestyle{fancy} %开始使用页眉页脚
\setcounter{page}{1} %论文页码从正文开始记数

\input{./body/SectionA.tex} %正文第一章
\input{./body/SectionB.tex} %正文第二章
\input{./body/SectionC.tex} %正文第三章
\input{./body/SectionD.tex} %正文第四章
\input{./body/SectionE.tex} %正文第五章

\theendnotes %尾注(若没有尾注请将本行删除)
\ClearPageStyle

%生成参考文献
\phantomsection
\addcontentsline{toc}{section}{参考文献}
\printbibliography[title={\centerline{\bfseries\sffamily \zihao {-3}参考文献}}]
\ClearPageStyle

%生成附录
\phantomsection
\addtocontents{toc}{\setcounter{tocdepth}{1}}
\addcontentsline{toc}{section}{附录}
\setcounter{subsection}{0}
\ctexset { subsection = { name={,},number={\arabic{subsection}},format={\rmfamily \zihao {5}} } }
\ctexset { subparagraph = { name={(,)},number={\arabic{subparagraph}},format={\rmfamily \zihao {5}},indent=2em } }
\section*{\centerline{附录}}

\subsection{实验数据}
\subparagraph{吐槽}
2019年的样板做得实在太烂了

\subsection{调查结果}
23333333333333333333333333333333333333333
\ClearPageStyle

\makeacknowledgement %生成感谢

\end{document} 
