\usepackage[thmmarks,hyperref]{ntheorem} %定义命令环境使用的宏包
\usepackage[heading,zihao=-4]{ctex} %用来提供中文支持
\usepackage{amsmath,amssymb} %数学符号等相关宏包
\usepackage{graphicx} %插入图片所需宏包
\usepackage{xspace} %提供一些好用的空格命令
\usepackage{tikz-cd} %画交换图需要的宏包
\usepackage{url} %更好的超链接显示
\usepackage{array,booktabs} %表格相关的宏包
\usepackage{caption} %实现图片的多行说明
\usepackage{float} %图片与表格的更好排版
\usepackage{ulem} %更好的下划线

\usepackage[top=2.5cm, bottom=2.0cm, left=3.0cm, right=2.0cm]{geometry} %设置页边距

\usepackage{fontspec} %设置字体需要的宏包

%设置西文字体为Times New Roman,如果没有则以开源近似字体代替
\IfFontExistsTF{Times New Roman}{
	\setmainfont{Times New Roman}
}{
	\usepackage{newtxtext,newtxmath}
}

%设置文档中文字体。优先次序:中易 > Adobe > 华文(Mac) > Fandol
\IfFontExistsTF{SimSun}{
	\setCJKmainfont[AutoFakeBold=2,ItalicFont=KaiTi]{SimSun}
}{
	\IfFontExistsTF{AdobeSongStd-Light}{
		\setCJKmainfont[AutoFakeBold=2,ItalicFont=AdobeKaitiStd-Regular]{AdobeSongStd-Light}
	}{
		\IfFontExistsTF{STSong}{
			\setCJKmainfont[AutoFakeBold=2,BoldFont=STHeiti,ItalicFont=STKaiti]{STSong}
		}{
			\setCJKmainfont[AutoFakeBold=2,ItalicFont=FandolKai-Regular]{FandolSong-Regular}
		}
	}
}
\IfFontExistsTF{SimHei}{
	\setCJKsansfont[AutoFakeBold=2]{SimHei}
}{
	\IfFontExistsTF{AdobeHeitiStd-Regular}{
		\setCJKsansfont[AutoFakeBold=2]{AdobeHeitiStd-Regular}
	}{
		\IfFontExistsTF{STHeiti}{
			\setCJKsansfont [AutoFakeBold=2]{STHeiti}
		}{
			\setCJKsansfont[AutoFakeBold=2]{FandolHei-Regular}
		}
	}
}


\IfFileExists{zhlineskip.sty}{
	%Microsoft Word 样式的1.5倍行距(按中易字体计算)
	\usepackage[
		restoremathleading=false,
		UseMSWordMultipleLineSpacing,
		MSWordLineSpacingMultiple=1.5
	]{zhlineskip}
}{
	\linespread{1.621} %1.5倍行距
}

\showboxdepth=5
\showboxbreadth=5

%设置各级系统的编号格式
\setcounter{secnumdepth}{5}
\ctexset { section = { name={,、},number={\chinese{section}},format={\centering \sffamily \zihao {-4}} } }
\ctexset { subsection = { name={(,)},number={\chinese{subsection}},format={\centering \sffamily \zihao {-4}} } }
\ctexset { subsubsection = { name={,.},number={\arabic{subsubsection}},format={\sffamily \zihao {-4}} } }
\ctexset { paragraph = { name={(,)},number={\arabic{paragraph}},format={\sffamily \zihao {-4}} } }
\ctexset { subparagraph = { name={,)},number={\arabic{subparagraph}},format={\sffamily \zihao {-4}} } }

\usepackage[bottom,perpage]{footmisc}               %脚注,显示在每页底部,编号按页重置
\renewcommand*{\footnotelayout}{\zihao{-5}\rmfamily}  %设置脚注为小五号宋体
\renewcommand{\thefootnote}{[\arabic{footnote}]}    %设置脚注标记为  [编号]

%设置页眉页脚
\usepackage{fancyhdr}
\lhead{华东师范大学学士学位论文}
\chead{}
\rhead{\TitleCHS}
\lfoot{}
\cfoot{\thepage}
\rfoot{}

\usepackage{xcolor} %彩色的文字

\usepackage[hidelinks]{hyperref} %各种超链接必备
\usepackage{cleveref} %交叉引用

%设置尾注
\usepackage{endnotes}
\renewcommand{\enotesize}{\zihao{-5}}
\renewcommand{\notesname}{\sffamily \zihao {-4} 尾注}
\renewcommand\enoteformat{
	\raggedright
	\leftskip=1.8em
	\makebox[0pt][r]{\theenmark. \rule{0pt}{\dimexpr\ht\strutbox+\baselineskip}}
}
\renewcommand\makeenmark{\textsuperscript{[尾注\theenmark]}}
\usepackage{footnotebackref}

%定义证明与解环境
\theoremstyle{nonumberplain}
\theorembodyfont{\upshape}
\theoremseparator{:}
\theoremsymbol{\ensuremath{\square}}
\newtheorem{proof}{\bfseries \sffamily 证明}
\theoremsymbol{\ensuremath{\blacksquare}}
\newtheorem{solution}{\bfseries \sffamily 解}

%定义各种常用环境
\theoremstyle{plain}
\theoremseparator{.}
\theorembodyfont{\upshape}
\theoremsymbol{}
\newtheorem{theorem}{\bfseries \sffamily 定理}[section]
\renewtheorem*{theorem*}{\bfseries \sffamily 定理}
\newtheorem{lemma}[theorem]{\bfseries \sffamily 引理}
\renewtheorem*{lemma*}{\bfseries \sffamily 引理}
\newtheorem{corollary}[theorem]{\bfseries \sffamily 推论}
\renewtheorem*{corollary*}{\bfseries \sffamily 推论}
\newtheorem{definition}[theorem]{\bfseries \sffamily 定义}
\renewtheorem*{definition*}{\bfseries \sffamily 定义}
\newtheorem{conjecture}[theorem]{\bfseries \sffamily 猜想}
\renewtheorem*{conjecture*}{\bfseries \sffamily 猜想}
\newtheorem{problem}[theorem]{\bfseries \sffamily 问题}
\renewtheorem*{problem*}{\bfseries \sffamily 问题}
\newtheorem{proposition}[theorem]{\bfseries \sffamily 命题}
\renewtheorem*{proposition*}{\bfseries \sffamily 命题}
\newtheorem{remark}[theorem]{\bfseries \sffamily 注记}
\renewtheorem*{remark*}{\bfseries \sffamily 注记}
\newtheorem{example}[theorem]{\bfseries \sffamily 例}
\renewtheorem*{example*}{\bfseries \sffamily 例}

%设置各种常用环境的交叉引用格式
\crefformat{theorem}{#2\bfseries{\sffamily 定理} #1#3}
\crefformat{lemma}{#2\bfseries{\sffamily 引理} #1#3}
\crefformat{corollary}{#2\bfseries{\sffamily 推论} #1#3}
\crefformat{definition}{#2\bfseries{\sffamily 定义} #1#3}
\crefformat{conjecture}{#2\bfseries{\sffamily 猜想} #1#3}
\crefformat{problem}{#2\bfseries{\sffamily 问题} #1#3}
\crefformat{proposition}{#2\bfseries{\sffamily 命题} #1#3}
\crefformat{remark}{#2\bfseries{\sffamily 注记} #1#3}
\crefformat{example}{#2\bfseries{\sffamily 例} #1#3}

%允许公式跨页显示
\allowdisplaybreaks

%屏蔽无关的Warning
\usepackage{silence}
\WarningFilter*{biblatex}{Conflicting options.\MessageBreak'eventdate=iso' requires 'seconds=true'.\MessageBreak Setting 'seconds=true'}

%使用biblatex管理文献,输出格式使用gb7714-2015标准,后端为biber
\usepackage[backend=biber,style=gb7714-2015,hyperref=true]{biblatex}

%提供了附录支持并显示在目录中
\usepackage[titletoc,title]{appendix}
\renewcommand{\appendixtocname}{附录}

%重定义生成附录的命令,使得每个附录都单独成页
\newcommand{\apdx}[1] {
	\clearpage
	\section{#1}}

%生成附录,请勿改动
\newcommand{\makeapdx}{
	\IfFileExists{./ending/Appendix.tex}{
	\clearpage
	\begin{appendices}
		\renewcommand{\thesection}{\chinese{section}、}
		\section*{\centerline{附录}}

\subsection{实验数据}
\subparagraph{吐槽}
2019年的样板做得实在太烂了

\subsection{调查结果}
23333333333333333333333333333333333333333
	\end{appendices}
	}{}
}

%生成感谢,请勿改动
\newcommand{\makeacknowledgement}{
	\clearpage
	\input{./ending/acknowledgement.tex}
}

%For Algorithm
\usepackage{algorithm,algorithmicx,algpseudocode}
\floatname{algorithm}{算法}
\renewcommand{\algorithmicrequire}{\textbf{输入:}}
\renewcommand{\algorithmicensure}{\textbf{输出:}}

%使表格中的脚注也能够正常显示
%\usepackage{footnote}
%\makesavenoteenv{table}

%可能会需要在用自然语言描述算法步骤时使用的宏包
\usepackage{enumitem}

%表格单元格内换行
\newcommand{\tabincell}[2]{\begin{tabular}{@{}#1@{}}#2\end{tabular}}

%显示 1、2级标题
\setcounter{tocdepth}{2}

%设置目录字体
\usepackage{tocloft}
\renewcommand{\contentsname}{\hfill \bfseries \sffamily \zihao{-4} 目\hspace*{2em}录 \hfill}
\renewcommand{\cftaftertoctitle}{\hfill}
\renewcommand{\cfttoctitlefont}{\sffamily}
\renewcommand{\cftsubsubsecfont}{\sffamily}
\renewcommand{\cftsubsecfont}{\sffamily}
\renewcommand{\cftsecfont}{\bfseries \sffamily}
\renewcommand{\cftsecleader}{\cftdotfill{\cftdotsep}}

%灵活的行距定义(用于封面)
\usepackage{setspace}

%提供封面标题的自动缩放
\usepackage[many]{tcolorbox}
\tcbset{
  colframe=white,
  colback=white,
  size=tight,
  nobeforeafter,
  valign=center,
  fit fontsize macros,
  fit algorithm=areasize
}
